Model transformations are critical points in the Model-driven Engineering (MDE) development process. The quality of the resulting system is highly influenced by the quality of the employed model transformations to produce the systems. However, users of transformations have to deal with the problem that transformations are difficult to debug and test for correctness. Such tests require a specification
%Checking this property and the verification must be done conforming to an specification
which expresses what is correct and what is not, something that is currently not supported by most transformation languages.
% but nowadays, the existing transformation languages does not let to define it.

A possible solution is to define with specification languages the requirements that a transformation has to fulfil. There are several approaches available for defining constraints on the input and output models as well as on the relationships between them (for an overview see \cite{VallecilloGBWH12}). These constraints are used as a blueprint for developing the model transformations employing implementation languages such as ATL, QVT, or graph transformations. Thus, the specification and the implementations are normally not coupled at all, which has several advantages but may also lead to disadvantages. In particular, when it comes to tracking errors, the missing traceability between specifications and implementations hampers the debugging process. Often the specifications are employed as oracles to check the transformation result. In case constraints are not fulfilled, the elements involved in the constraint evaluation may give a valuable information for the transformation engineer, but the link to the transformation rules is not available.

To tackle this limitation, we present in this paper a first solution for measuring the alignment between a constraint and a model transformation rule by applying an automated matching function. In particular, three different measures are introduced which provide different viewpoints on the alignment problem. We employ the approach for a case study and finally discuss how this general approach may be applied for specific use cases in model transformation engineering.

This paper is organized as follows. After this introduction, Section \ref{sec:Background} shows a brief description of model transformations, Section \ref{sec:MatchingTables} presents our proposal and Section \ref{sec:Evaluation} discusses the results. Then, Section \ref{sec:RelatedWork} presents related works and Section \ref{sec:NextSteps} gives some ideas about future research lines.
