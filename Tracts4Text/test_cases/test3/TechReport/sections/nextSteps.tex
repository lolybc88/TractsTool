The main motivation for this work was the need to track transformation rules that can be considered ``guilty'' for violating parts of the transformation specifications. Due to the generic nature of the matching tables, a multitude of further use cases emerge.

\textbf{Properties of Alignments.} Based on the matching tables, we are able to reason on the degree of tangling and scattering between constraints and rules. Scattering occurs when a single constraint is scattered across multiple rules, while tangling occurs when a single transformation rule is implementing multiple constraints at once. The work of Berg et al.~\cite{BergCH07} may be valid input to reason about design guidelines of transformation specifications and implementations based on matching tables.

\textbf{Refinement of Alignments.} More information may be extracted from constraints and transformation rules. For example, from the ATL transformations, inheritance between transformation rules, lazy rule calls, and types used in filters and bindings may be extracted. From the constraints, the accessed metamodel features may be extracted, too. Based on this additional information, more refined alignments may be explored. Furthermore, as we have mentioned in the evaluation, some constraints are using a multitude of types. To distinguish between types, e.g., types only required to navigate to the most relevant information in a model, types occurring more often in constraints may have less impact on the alignments as types that do not as frequently occur.

\textbf{Alignment-based Slicing.} Another direction for future work is to slice model transformations, metamodels, and models based on constraints. This is of course useful for debugging model transformations, however, using slicing techniques may be also beneficial for maintenance tasks. Imagine the requirements are changed by modifying a specific constraint. Adapting the transformation implementation to this change may be easier by reasoning only on a particular slice of the transformation problem referring to a subset of the transformation, metamodel, and models.
