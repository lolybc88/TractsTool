\documentclass{llncs}

%% Put edit comments in a really ugly standout display
\usepackage{ifthen}
\usepackage{amssymb}
\usepackage{color}
\usepackage[gray,yellow]{xcolor}
\usepackage{AMMALanguages}
\usepackage{hyperref}
\usepackage{hyperref}
\usepackage{graphicx}
\usepackage{amsmath}
\usepackage[lined,linesnumbered]{algorithm2e}

\setcounter{totalnumber}{5}

\newboolean{showcomments}
\setboolean{showcomments}{true} % toggle to show or hide comments
\ifthenelse{\boolean{showcomments}}
  {\newcommand{\nb}[2]{
    \fcolorbox{gray}{yellow}{\bfseries\sffamily\scriptsize#1}
    {\sf\small$\blacktriangleright$\textit{#2}$\blacktriangleleft$}
   }
   \newcommand{\version}{\emph{\scriptsize$-$working$-$}}
  }
  {\newcommand{\nb}[2]{}
   \newcommand{\version}{}
  }

\newcommand\antonio[1]{\nb{Antonio}{#1}}
\newcommand\manuel[1]{\nb{Manuel}{#1}}
\newcommand\loli[1]{\nb{Loli}{#1}}

\begin{document}

\title{Towards Tracking ``Guilty'' Transformation Rules}
\subtitle{A Requirements Perspective}

\author{ Loli Burgue\~no \and Manuel Wimmer \and Antonio Vallecillo }

\tocauthor{
 Antonio Vallecillo (Universidad de M\'alaga)
}

 \institute{
 GISUM/Atenea Research Group, Universidad de  M\'alaga, Spain
 \email{\{loli,mw,av\}@lcc.uma.es}
 }

\maketitle
\begin{abstract}
Several approaches for specifying the requirements for model transformations have been proposed. Most of them define constraints on source and target models as well as on the relationships between them. A major advantage of these approaches is their independence from transformation implementation languages and transformation implementations. However, when these constraints are used for testing, identifying the model transformation rules that violate the constraints is not possible. In this paper we present an approach for automatically aligning specifications of model transformations and their implementations. Matching functions establish these alignments based on the used metamodel elements in the constraints and rules. We present our first results and outline further use cases where an alignment between constraints and rules is beneficial.
\end{abstract}

%% A category with the (minimum) three required fields
%\category{H.4}{Information Systems Applications}{Miscellaneous}
%%A category including the fourth, optional field follows...
%\category{D.2.8}{Software Engineering}{Metrics}[complexity measures, performance measures]

%\terms{Theory}

%\keywords{Model } % NOT required for Proceedings

\section{Introduction}
\label{sec:Introduction}
Model transformations are critical points in the Model-driven Engineering (MDE) development process. The quality of the resulting system is highly influenced by the quality of the employed model transformations to produce the systems. However, users of transformations have to deal with the problem that transformations are difficult to debug and test for correctness. Such tests require a specification
%Checking this property and the verification must be done conforming to an specification
which expresses what is correct and what is not, something that is currently not supported by most transformation languages.
% but nowadays, the existing transformation languages does not let to define it.

A possible solution is to define with specification languages the requirements that a transformation has to fulfil. There are several approaches available for defining constraints on the input and output models as well as on the relationships between them (for an overview see \cite{VallecilloGBWH12}). These constraints are used as a blueprint for developing the model transformations employing implementation languages such as ATL, QVT, or graph transformations. Thus, the specification and the implementations are normally not coupled at all, which has several advantages but may also lead to disadvantages. In particular, when it comes to tracking errors, the missing traceability between specifications and implementations hampers the debugging process. Often the specifications are employed as oracles to check the transformation result. In case constraints are not fulfilled, the elements involved in the constraint evaluation may give a valuable information for the transformation engineer, but the link to the transformation rules is not available.

To tackle this limitation, we present in this paper a first solution for measuring the alignment between a constraint and a model transformation rule by applying an automated matching function. In particular, three different measures are introduced which provide different viewpoints on the alignment problem. We employ the approach for a case study and finally discuss how this general approach may be applied for specific use cases in model transformation engineering.

This paper is organized as follows. After this introduction, Section \ref{sec:Background} shows a brief description of model transformations, Section \ref{sec:MatchingTables} presents our proposal and Section \ref{sec:Evaluation} discusses the results. Then, Section \ref{sec:RelatedWork} presents related works and Section \ref{sec:NextSteps} gives some ideas about future research lines.



%\section{Background: Specification and Implementation of Model Transformations}
\section{Background}
\label{sec:Background}
In this section, we shortly introduce the formalisms used in this paper for specifying and implementing model transformations. As we shall see, these formalisms are not integrated, and thus, the developed artifacts are completely independent of each other.

\subsection{Specifying Transformations with Tracts}

One of the advantages of contracts is that they allow defining \emph{what} a piece of software does but not \emph{how} it is done. Contracts can be used to precisely specify the constraints (going beyond metamodel constraints) to be
satisfied by source models such that the transformation is applicable, i.e.,
\emph{preconditions} of the transformations. They can also be used to
express constraints on the target models, i.e., \emph{postconditions} of the
transformation. And finally, they can be used to specify constraints that need to
be satisfied by any pair of source/target models of a correct transformation, i.e., \emph{invariants} of the transformation.
Thus, a specification language should allow to formulate these three kinds of
contracts.

In literature there are several approaches which try to deal with the specification. Some of them work in the model level (\cite{Kolovos06},\cite{Lin04modelcomparison:}, \cite{Lin2005}, \cite{Garcia-Dominguez11}, \cite{Mottu2008}) and others in the metamodel level (\cite{BaloghBCGHMPPRVa10}, \cite{Guerra12}, \cite{Baudry06}, \cite{Anastasakis07}) but the main problem is the complexity. The specifications of a model
transformation can become monstrously large as far as the transformation is not
trivial (even far more complex than the transformation itself). The reasons
are, among others, the lack of modularity, having to deal with too many details
at the same time, and the excessive size. Because the specifications try to
capture all the model transformation behaviour in one huge set of constraints,
they become hard to write, debug and maintain. In addition, tests become quite
cumbersome, very complex, and computationally prohibitive to prove.

In order to deal with these problems, tracts were introduced in~\cite{TRACTS11} as a specification and black-box testing mechanism for model transformations.
Tracts provide modular pieces of specification, each one focusing on a particular transformation scenario. Thus every model transformation can be specified by means of a set of tracts, each one covering a particular use
case---which is defined in terms of particular input and output models and how
they should be related by the transformation. In this way, tracts allow
partitioning the full input space of the transformation into smaller, more focused
behavioural units, and to define specific tests for them. Basically, what we do
with the tracts is to identify the scenarios of interest to the user of the
transformation (each one defined by a tract) and check
whether the transformation behaves as expected in these scenarios.
Another characteristic of Tracts is that we do not require complete proofs,
just to check that the transformation works for the tract test suites, hence
providing a \emph{light-weight} form of verification.

In a nutshell, a tract defines a set of constraints on the \emph{source} and
\emph{target} metamodels, a set of \emph{source-target} constraints, and a
tract \emph{test suite}, i.e., a collection of source models satisfying the
source constraints. The constraints serve as ``contracts'' (in the sense of
contract-based design~\cite{Meyer92}) for the transformation in some particular
scenarios, and are expressed by means of OCL invariants. They provide the
\emph{specification} of the transformation. Figure~\ref{fig:TMT}
gives an overview on the used concepts and their connection.

Additionally, every tract provides a \emph{test suite} that allows to
operationalize the conformance tests. We do not provide the full behavioral
specification of a model transformation, but just a set of tracts that defines
how the transformation should behave in certain particular scenarios (or use
cases) which are the ones of interest to the user. We do not care how the
transformation works in the rest of the cases.

\begin{figure}[t]
\centering
\includegraphics[width=1\columnwidth]{images/tract_concepts}
\caption{Concepts in a Tract}
\label{fig:TMT}
\end{figure}

\begin{figure}[t]
\centering
\includegraphics[width=1\columnwidth]{images/Family2Person}
\caption{The Family and Person metamodels.}
\label{fig:Family2PersonsMM}
\end{figure}

For demonstrating how to use Tracts, we introduce the simple transformation example \emph{Families2Persons}\footnote{The complete example is available at \texttt{http://atenea.lcc.uma.es/index.php/Main\_Page/Resources\\/Tracts-ATL}}. The source and target metamodels of this transformation are shown in Figure \ref{fig:Family2PersonsMM}. For this example, a set of tracts is developed to consider only those families which count exactly four members (mother, father, daughter, son):

\begin{lstlisting}[numbers=none]
-- C1: SRC_oneDaughterOneSon
Family.allInstances->forAll(f|f.daughters->size=1 and f.sons->size=1)
		
-- C2: SRC_TRG_Mother2Female
Family.allInstances->forAll(fam|Female.allInstances->exists(f|fam.mother.firstName.concat(' ').concat(fam.lastName)=f.fullName))
		
-- C3: SRC_TRG_Daughter2Female
Family.allInstances->forAll(fam|Female.allInstances->exists(f|fam.daughters->exists(d|d.firstName.concat(' ').concat(fam.lastName)=f.fullName)))
		
-- C4: SRC_TRG_FatherSon2Male
Family.allInstances->forAll(fam|Male.allInstances->exists(m| fam.father.firstName.concat(' ').concat(fam.lastName)=m.fullName xor fam.sons->exists(s|m.firstName.concat(' ').concat(fam.lastName)=s.fullName))
		
-- C5: SRC_TRG_Female2MotherDaughter
Female.allInstances->forAll(f|Family.allInstances->exists(fam|fam.mother.firstName.concat(' ').concat(fam.lastName)=f.fullName xor fam.daughters->exists(d|d.firstName.concat(' ').concat(fam.lastName)=f.fullName)))
		
-- C6: SRC_TRG_Male2FatherSon
Male.allInstances->forAll(m|Family.allInstances->exists(fam|fam.father.firstName.concat(' ').concat(fam.lastName)=m.fullName xor fam.sons->exists(s|s.firstName.concat(' ').concat(fam.lastName)=m.fullName)))
		
-- C7: SRC_TRG_MemberSize_EQ_PersonSize
Member.allInstances->size=Person.allInstances->size

-- C8: TRG_PersonHasName
Person.allInstances->forAll(p|p.fullName <> '' and not p.fullName.oclIsUndefined())
\end{lstlisting}

\subsection{Implementing Transformations with ATL}

Given this specification, a model transformation language may be selected to implement the transformation. The ATLAS Transformation Language (ATL) \cite{ATL} is a common choice. ATL is designed as a hybrid model transformation language containing a mixture of declarative and imperative constructs for defining uni-directional transformations. An ATL transformation is mainly composed by a set of \emph{rules}. A rule describes how a subset of the target model should be generated from a subset of the source model. A rule consists of an \textit{input} pattern (having an optional \emph{filter} condition) which is matched on the source model and an \textit{output} pattern which produces certain elements in the target model for each match of the input pattern. The values of the target model elements are assigned in \emph{bindings} which calculate the values by OCL expressions. Given the metamodels in Figure \ref{fig:Family2PersonsMM} and the tracts, a possible implementation in ATL may be as follows:

\begin{lstlisting}[numbers=none]
module Families2Persons;
create OUT: Persons from IN: Families;

helper context Families!Member def: isFemale(): Boolean =
    if not self.familyMother.oclIsUndefined() then
		true
	else
		if not self.familyDaughter.oclIsUndefined() then
			true
		else
			false
		endif
	endif;

helper context Families!Member def: familyName: String =
	if not self.familyFather.oclIsUndefined() then
		self.familyFather.lastName
	else
		if not self.familyMother.oclIsUndefined() then
			self.familyMother.lastName
		else
			if not self.familySon.oclIsUndefined() then
				self.familySon.lastName
			else
				self.familyDaughter.lastName
			endif
		endif
	endif;

rule Member2Male { -- R1 for short
  from
    s: Families!Member (not s.isFemale())
  to
    t: Persons!Male (fullName <- s.firstName + ' ' + s.familyName )
}

rule Member2Female { -- R2 for short
  from
    s: Families!Member (s.isFemale())
  to
    t: Persons!Female (fullName <- s.firstName + ' ' + s.familyName)
}
\end{lstlisting}

%  if not self.familyFather.oclIsUndefined() then self.familyFather.lastName
%  else ...
%    if not self.familyMother.oclIsUndefined() then self.familyMother.lastName
%    else
%      if not self.familySon.oclIsUndefined() then self.familySon.lastName
%      else
%         self.familyDaughter.lastName
%      endif
%    endif
%  endif;



\section{Matching Constraints and Rules: Basics}
\label{sec:MatchingTables}
As we have seen in the previous section, Tracts allow defining constraints for transformations, while ATL uses rules to express model transformations. Having independent artifacts for the specification and implementation allows for freedom which formalisms to choose for both levels and how to implement the specifications. However, the following questions cannot be answered without through reasoning on both artifact types: $(a)$  Which transformation rule(s) implement(s) which constraint(s)? $(b)$ Are all constraints covered by transformation rules? and $(c)$ Are all transformation rules covered by constraints?

\subsection{Challenges}

In general, we there are two possibilities to compute alignments between rules and constraints to answer the previous questions. First, there is \emph{static alignment} by reasoning only on the constraints and transformation rules without executing them, and second, there is the possibility to \emph{dynamically} explore the relationships by running the transformations as well as checking the constraints to find overlaps on accessed model elements.

While the second approach may lead to more precise alignments, the alignments are always specific to a given input model. If a more general alignment should be derived, the static approach would more beneficial. However, static alignment seems to be more challenging, because there is a complete paradigm mismatch of the specification language and the implementation language. While in Tracts general OCL expressions are used, in ATL the prime elements are rules. Thus, current generic model matching frameworks cannot be employed, because they produce matches based on structural equivalences. But in our case we have two different languages following different programming paradigms. Thus no structural equivalences are identifiable in a generic manner and other means for comparison have to be found.

The common denominator of constraints and rules are the metamodel elements used, which may give an indication of the relatedness. Therefore, we describe next how this information can be obtained from constraints and rules, compare the extracted information, and present the results to the user.

\subsection{Matching Tables: 3 different Viewpoints}
\label{subsec:MatchingTables}

For representing the alignments between constraints and rules, we use tabular representations which we call \emph{matching tables}. Our aim is to automatically compute such tabular representations by employing matching functions and to provide different viewpoints on the alignments found. Using different viewpoints on alignments supports answering different questions as outlined in~\cite{WongGH00}.

Given a set of constraints and a set of rules, the corresponding matching tables are computed based on the types of the elements, i.e., the classes from the metamodels, that they contain. In these tables, each cell links a constraint and a rule with a  specific value between 0 and 1. Let $C_{i}$ be the set of types extracted for constraint $i$ and $R_{j}$ for rule $j$. In the following, three different metrics are introduced that provide different viewpoints on the types overlaps.

The \emph{constraint coverage} (CC) metric states the coverage for constraint $i$ by a given rule $j$. For this metric, the value for the cell $[i,j]$ is given by the following formula.

\begin{align}
CC_{i,j} = \dfrac{| C_{i} \cap R_{j} |}{| C_{i} |}
\label{for:firstmetric}
\end{align}

As the denominator is the number of types in $C_{i}$, the result is relative to constraint $i$ and we interpret this value for rule traceability, i.e., to find rules which are related to the given~constraint.

The \emph{rule coverage} (RC) metric states the coverage for rule $j$ by a given constraint $i$. We use this value to express constraint traceability, i.e., to find the constraints most closely related to a given rule. The following formula is used to compute the values for this metric.

\begin{align}
RC_{i,j} = \dfrac{| C_{i} \cap R_{j} |}{| R_{j} |}
\label{for:secondmetric}
\end{align}



The last metric is relative to both constraints and rules. Thus, it gives a statement of the relatedness of both without defining a direction for interpreting the values. The  \emph{relatedness} of \emph{constraints} and \emph{rules} (RCR) metric is as follows.

\begin{align}
RCR_{i,j} = \dfrac{| C_{i} \cap R_{j} |}{| C_{i} \cup R_{j} |}
\label{for:thirdmetric}
\end{align}

After extracting the types for constraints and rules, there exist five possible cases, as depicted in Figure~\ref{fig:Sectorial} using Venn diagrams. Let us study each one and comment some of the particular properties of these metrics.


\begin{figure}[h!]
\centering
\includegraphics[width=200pt]{images/Sectorial}
\caption{Possible situations for $C_{i}$ and $R_{j}$.}
\label{fig:Sectorial}
\end{figure}


In case (a), every constraint type is contained by the set of rule types, $C_{i} \subseteq R_{j}$, thus the value for the CC metric is 1 and it means that the constraint is fully covered by the rule. The other metrics have a value lower than 1.

Case (b) is the opposite to case (a), $R_{j} \subseteq C_{i}$, and here the RC metric is always 1. One possible interpretation follows. If after the transformation execution and constraint verification we detect that $C_{i}$ fails, we know that the failure probably comes from $R_{j}$ or a part of it and the bigger the value of RC is, the most likely it is that the failure comes from $R_{j}$.

For case (c), $C_{i}$ and $R_{j}$ are disjoint sets. Thus all the metrics are 0 which means that the given constraint and the given rule are completely independent.

In case (d), every metric will have a value between 0 and 1. The exact value will be dependent on the size of the sets and the number of common elements. For example, the bigger the common part for $C_{i}$ is, the closer the value for metric CC will be to 1. The lower the common part is, the closer CC will be to 0. It is the same with $R_{j}$ and metric RC. Considering the third metric in case (d), its value depends only on the size of the common part. Thus, the bigger it is, the closer the value is to 1.

In case (e), both types of constraints and rules are the same set, consequently each metric is 1. It is the situation where a constraint and a rule are totally covered by each other.


\begin{table}[t]
\centering
\caption{Used types for Families2Person example.}
\begin{tabular}{|c|l|} \hline
Constraint/Rule&Involved Types\\ \hline
C1&Member, Family\\ \hline
C2&Member, Family, Female\\ \hline
C3&Member, Family, Female\\ \hline
C4&Member, Family, Male\\ \hline
C5&Member, Family, Female\\ \hline
C6&Member, Family, Male\\ \hline
C7&Member, Person\\ \hline
C8&Person\\ \hline
R1&Member, Male\\ \hline
R2&Member, Female\\ \hline
\end{tabular}
\label{tab:Family2PersonExample}
\end{table}

The \emph{Families2Persons} example presented in the previous section counts on two rules and eight constraints. The types used in the constraints and rules are summarized in Table \ref{tab:Family2PersonExample}.
According to the types extracted for this example, the corresponding matching tables are shown in Table~\ref{tab:TableSM2LUT}. The second and third columns express the constraint coverage, the fourth and fifth the rule coverage, and the sixth and seventh the relatedness. Please note that this is a small example with the only intention of showing how the metrics are computed. Section \ref{sec:Evaluation} shows matching tables for a larger example as well as their interpretation.

\begin{table}[t]
\centering
\caption{Families2Person matching tables.}
\begin{tabular}{c|c|c||c|c||c|c|}
\cline{2-7}
& \multicolumn{2}{|c||}{CC} & \multicolumn{2}{|c||}{RC} & \multicolumn{2}{|c|}{RCR}\\ \cline{2-7}
&R1&R2&R1&R2&R1&R2\\ \hline
\multicolumn{1}{|c|}{C1}&0.5&0.5&0.5&0.5&0.33&0.33\\ \hline
\multicolumn{1}{|c|}{C2}&0.33&0.66&0.5&1&0.25&0.66\\ \hline
\multicolumn{1}{|c|}{C3}&0.33&0.66&0.5&1&0.25&0.66\\ \hline
\multicolumn{1}{|c|}{C4}&0.66&0.33&1&0.5&0.66&0.25\\ \hline
\multicolumn{1}{|c|}{C5}&0.33&0.66&0.5&1&0.25&0.66\\ \hline
\multicolumn{1}{|c|}{C6}&0.66&0.33&1&0.5&0.66&0.25\\ \hline
\multicolumn{1}{|c|}{C7}&0.5&0.5&0.5&0.5&0.33&0.33\\ \hline
\multicolumn{1}{|c|}{C8}&0.0&0.0&0.0&0.0&0.33&0.33\\ \hline
\end{tabular}
\label{tab:TableSM2LUT}
\end{table}





\section{Variations of Matching Function}
\label{sec:Evaluation}

Tracts also provide mechanisms for testing model transformations. But when a tract fails, it is useful to know which rule or rules are responsible for the failure. This need originated the work we are presenting in this paper. Finding the ``guilty'' rules is supported by rule traceability (CC metric), and complemented by the RCR metric that reflects additional information. As we presented in the previous section, we count on several metrics which allows us to reason on more complicated cases.

\subsection{Transformation Example: UML2ER}

The \emph{Families2Persons} example is a rather small example, but sufficient for demonstrating the basic process of computing the different metrics. Of course, the usefulness of the process described above is intended to show up for larger examples having a comprehensive set of constraints and rules. In comparison with the \emph{Families2Persons} example, the following \emph{UML2ER} example is larger. This transformation scenario  considers the metamodels for a simplified version of UML class diagrams and Entity-Relationship diagrams. The metamodels are illustrated in Figure~\ref{fig:UML2ER}.

\begin{figure}[t]
\centering
\includegraphics[width=1\textwidth]{images/UML2ER}
\caption{The UML and ER metamodels.}
\label{fig:UML2ER}
\end{figure}

The specification of the transformation comprises eight source-target constraints where two kinds of constraints are used. One kind is comparing the amount of instances for a given source and target class, while the other kind is checking for equivalency of elements based on containment relationships and value correspondences.

\begin{lstlisting}[numbers=none]

-- SRC_TRG_Package2ERModel
-- C1: Package.allInstances->forAll(p | ERModel.allInstances->one(e | p.name = e.name))

-- SRC_TRG_Package2ERModel and Class2EntityType and NESTING
-- C2: Package.allInstances->forAll(p | ERModel.allInstances->one(e | p.name = e.name and p.ownedElements-> forAll(class | e.entities->one(entity | entity.name = class.name))))

-- SRC_TRG_Package2ERModel and Class2EntityType and Property2Feature and NESTING
-- C3: Package.allInstances->forAll(p | ERModel.allInstances->one(e | p.name = e.name and p.ownedElements->forAll(class | e.entities->one(entity | entity.name = class.name and class.ownedProperty->forAll(prop | entity.features->forAll(f | f.name = prop.name))))))

-- SRC_TRG_NamedElement2Element
-- C4: NamedElement.allInstances-> size() = Element.allInstances->size()

-- SRC_TRG_Package2ERModel
-- C5: Package.allInstances->size() = ERModel.allInstances->size()

-- SRC_TRG_Class2EntityType
-- C6: Class.allInstances->size() = EntityType.allInstances->size()

-- SRC_TRG_Property2Feature
-- C7: Property.allInstances->size() = Feature.allInstances->size()

-- SRC_TRG_Package2ERModel and Class2EntityType and Property2Attribute and NESTING
-- C8: Package.allInstances->forAll(p | ERModel.allInstances->one(e | p.name = e.name and p.ownedElements->forAll(class | e.entities->one(entity | entity.name = class.name and class.ownedProperty->forAll(prop | prop.primitiveType<> "" and (not prop.primitiveType.oclIsUndefined()) implies entity.features-> select(f|f.oclIsTypeOf(Attribute)) -> one(f | f.name = prop.name))))))

-- SRC_TRG_Package2ERModel and Class2EntityType and Property2WeakReference and NESTING
-- C9: Package.allInstances->forAll(p | ERModel.allInstances->one(e | p.name = e.name andp.ownedElements->forAll(class | e.entities->one(entity | entity.name = class.name and class.ownedProperty->forAll(prop | prop.complexType <> null implies entity.features-> select(f|f.oclIsTypeOf(Reference)) -> one(f | f.name = prop.name and prop.isContainment impliesf.oclIsTypeOf(WeakReference)))))))

-- SRC_TRG_Package2ERModel and Class2EntityType and Property2StrongReference and NESTING
-- C10: Package.allInstances->forAll(p | ERModel.allInstances->one(e | p.name = e.name and p.ownedElements->forAll(class | e.entities->one(entity | entity.name = class.name and class.ownedProperty->forAll(prop | prop.complexType <> null implies entity.features-> select(f|f.oclIsTypeOf(Reference)) -> one(f | f.name = prop.name and not prop.isContainment impliesf.oclIsTypeOf(StrongReference)))))))
\end{lstlisting}

The transformation contains eight transformation rules, whereas three of the rules are abstract rules and a multitude of inheritance relationships between the rules exists: $R8, R7 < R6; R6, R5 < R4; R4, R3, R2 < R1$.

\begin{lstlisting}[numbers=none]
module UML2ER;
create OUT : ER from IN : SimpleUML;

rule NamedElement{ -- R1
	from
		s: SimpleUML!NamedElement
	to
		t: ER!Element (
			name <- s.name	
		)	
}

rule Package extends NamedElement{ -- R2
	from
		s: SimpleUML!Package
	to
		t: ER!ERModel (
			entities <- s.ownedElements	
		)
}

rule Class extends NamedElement { -- R3
	from s: SimpleUML!Class
	to t: ER!EntityType (
			features <- s.ownedProperty->collect (e | thisModule.Property(e))
    )
}


lazy abstract rule Property extends NamedElement{ -- R4
	from s: SimpleUML!Property
	to t: ER!Feature ()
}


lazy rule Attributes extends Property{ -- R5
	from s: SimpleUML!Property (not s.primitiveType.oclIsUndefined())
	to	t: ER!Attribute (
			type <- s.primitiveType	
	)
}

lazy abstract rule References extends Property { -- R6
	from s: SimpleUML!Property (not s.complexType.oclIsUndefined() )
	to t: ER!Reference (
			type <- s.complexType
	)
}

lazy rule WeakReferences extends References { -- R7
	from s: SimpleUML!Property (not s.isContainment)
	to t: ER!WeakReference
}

lazy rule StrongReferences extends References{ -- R8
	from s: SimpleUML!Property (s.isContainment)
	to t: ER!StrongReference
}
\end{lstlisting}

We now report on the results we got when applying our approach for the \emph{UML2ER} example. Tables \ref{tab:TableUML2ER4C}-\ref{tab:TableUML2ER4CR} illustrate the corresponding matching tables.

Now assume that the transformation is executed, the constraints are checked, and C7 fails. By looking at Table~\ref{tab:TableUML2ER4C}, we find a complete coverage of C7 by R4. Thus, it is more likely that the failure is due to R4, instead of coming from R5, R6, R7 or R8. In contrast, the cells R1/C7, R2/C7 and R3/C7 indicate that C7 is completely independent from R1, R2 and R3. Thus, it seems more appropriate in the error tracking process to start with R4 and then continuing with R5-R8.

Let us now suppose that a constraint fails and there are two or more cells in its row in Table \ref{tab:TableUML2ER4C} with value 1, while the value of these cells in the other two tables is different. In this case, we should choose to review the rule which has the higher value for RCR metric (Table \ref{tab:TableUML2ER4CR}).

\begin{table}[!h]
\centering
\caption{Matching table using CC metric.}
\begin{tabular}{c|c|c|c|c|c|c|c|c|c|} \cline{2-10}
&R1&R2&R3&R4&R5&R6&R7&R8&standard deviation\\ \hline
\multicolumn{1}{|c|}{C1}&0.0&1.0&0.0&0.0&0.0&0.0&0.0&0.0&0.330718914\\ \hline
\multicolumn{1}{|c|}{C2}&0.0&0.5&0.5&0.0&0.0&0.0&0.0&0.0&0.216506351\\ \hline
\multicolumn{1}{|c|}{C3}&0.0&0.33&0.33&0.33&0.16&0.16&0.16&0.16&0.110239638\\ \hline
\multicolumn{1}{|c|}{C4}&1.0&0.0&0.0&0.0&0.0&0.0&0.0&0.0&0.330718914\\ \hline
\multicolumn{1}{|c|}{C5}&0.0&1.0&0.0&0.0&0.0&0.0&0.0&0.0&0.330718914\\ \hline
\multicolumn{1}{|c|}{C6}&0.0&0.0&1.0&0.0&0.0&0.0&0.0&0.0&0.330718914\\ \hline
\multicolumn{1}{|c|}{C7}&0.0&0.0&0.0&1.0&0.5&0.5&0.5&0.5&0.330718914\\ \hline
\multicolumn{1}{|c|}{C8}&0.0&0.28&0.28&0.28&0.28&0.14&0.14&0.14&0.099424364\\ \hline
\multicolumn{1}{|c|}{C9}&0.0&0.25&0.25&0.25&0.12&0.25&0.25&0.12&0.088388348\\ \hline
\multicolumn{1}{|c|}{C10}&0.0&0.25&0.25&0.25&0.12&0.25&0.12&0.25&0.088388348\\ \hline
\end{tabular}
\label{tab:TableUML2ER4C}
\end{table}

\begin{table}[!h]
\centering
\caption{Matching table using RC metric.}
\begin{tabular}{c|c|c|c|c|c|c|c|c|c|} \cline{2-10}
&R1&R2&R3&R4&R5&R6&R7&R8&standard deviation\\ \hline
\multicolumn{1}{|c|}{C1}&0.0&1.0&0.0&0.0&0.0&0.0&0.0&0.0&0.330718914\\ \hline
\multicolumn{1}{|c|}{C2}&0.0&1.0&1.0&0.0&0.0&0.0&0.0&0.0&0.433012702\\ \hline
\multicolumn{1}{|c|}{C3}&0.0&1.0&1.0&1.0&0.5&0.5&0.5&0.5&0.330718914\\ \hline
\multicolumn{1}{|c|}{C4}&1.0&0.0&0.0&0.0&0.0&0.0&0.0&0.0&0.330718914\\ \hline
\multicolumn{1}{|c|}{C5}&0.0&1.0&0.0&0.0&0.0&0.0&0.0&0.0&0.330718914\\ \hline
\multicolumn{1}{|c|}{C6}&0.0&0.0&1.0&0.0&0.0&0.0&0.0&0.0&0.330718914\\ \hline
\multicolumn{1}{|c|}{C7}&0.0&0.0&0.0&1.0&0.5&0.5&0.5&0.5&0.330718914\\ \hline
\multicolumn{1}{|c|}{C8}&0.0&1.0&1.0&1.0&1.0&0.5&0.5&0.5&0.347985273\\ \hline
\multicolumn{1}{|c|}{C9}&0.0&1.0&1.0&1.0&0.5&1.0&1.0&0.5&0.353553391\\ \hline
\multicolumn{1}{|c|}{C10}&0.0&1.0&1.0&1.0&0.5&1.0&0.5&1.0&0.353553391\\ \hline
\end{tabular}
\label{tab:TableUML2ER4R}
\end{table}

\begin{table}[[!h]]
\centering
\caption{Matching table using RCR metric.}
\begin{tabular}{c|c|c|c|c|c|c|c|c|c|} \cline{2-10}
&R1&R2&R3&R4&R5&R6&R7&R8&standard deviation\\ \hline
\multicolumn{1}{|c|}{C1}&0.0&1.0&0.0&0.0&0.0&0.0&0.0&0.0&0.330718914\\ \hline
\multicolumn{1}{|c|}{C2}&0.0&0.5&0.5&0.0&0.0&0.0&0.0&0.0&0.216506351\\ \hline
\multicolumn{1}{|c|}{C3}&0.0&0.33&0.33&0.33&0.14&0.14&0.14&0.14&0.115267361\\ \hline
\multicolumn{1}{|c|}{C4}&1.0&0.0&0.0&0.0&0.0&0.0&0.0&0.0&0.330718914\\ \hline
\multicolumn{1}{|c|}{C5}&0.0&1.0&0.0&0.0&0.0&0.0&0.0&0.0&0.330718914\\ \hline
\multicolumn{1}{|c|}{C6}&0.0&0.0&1.0&0.0&0.0&0.0&0.0&0.0&0.330718914\\ \hline
\multicolumn{1}{|c|}{C7}&0.0&0.0&0.0&1.0&0.33&0.33&0.33&0.33&0.30900827\\ \hline
\multicolumn{1}{|c|}{C8}&0.0&0.28&0.28&0.28&0.28&0.12&0.12&0.12&0.10333158\\ \hline
\multicolumn{1}{|c|}{C9}&0.0&0.25&0.25&0.25&0.11&0.25&0.25&0.11&0.091009322\\ \hline
\multicolumn{1}{|c|}{C10}&0.0&0.25&0.25&0.25&0.11&0.25&0.11&0.25&0.091009322\\ \hline
\end{tabular}
\label{tab:TableUML2ER4CR}
\end{table}

Another situation happens when a rule is fully covered by a constraint. Consequently, the value of the corresponding cell in Table \ref{tab:TableUML2ER4R} is 1, but not 1 in the rest of the tables. For instance, in Table~\ref{tab:TableUML2ER4R}, C8 has value 1 for R2, R3, R4 and R5 but less than 1 in Tables \ref{tab:TableUML2ER4C} and \ref{tab:TableUML2ER4CR}. Again, in this situation it is better to chose the rule with a higher value in Table \ref{tab:TableUML2ER4CR}. Another interesting information that we can extract from Table \ref{tab:TableUML2ER4R} is whether each rule is fully covered by a set of Tracts or if further Tracts are needed to enhance the test coverage.
In the last case (although not shown in our example), a constraint which fails has values in all tables different from 1. In that case, one more time, we must review the rule with the higher value for Table~\ref{tab:TableUML2ER4CR}.

To sum up, what is the possibility of finding a rule for a failed constraint? As mentioned before, we have two kinds of constraints. For constraints checking that the amount of instances for source and target classes should be equal, unambiguous alignments can be found. For the other kind, the situation is different. Depending on the size of the constraints and the amount of used types, we may find several rules having similar alignment ratings. Thus, we explore in the following subsections several possibilities for revising the matching function to produce more distinct values.

\subsection{Considering Inheritance}

In this section, we propose a modification of the matching function in order to consider \emph{inheritance} between meta-classes. This is of major importance, because ATL transformation rules are matching for direct instances of the given input pattern types, but also for indirect instances, i.e., instances of subclasses. The same is true for tracts. If for instance the amount of instances of a certain type is computed (using the \emph{allInstances} operation), then not only the direct instances are counted but also the indirect instances.

What we exactly do to consider inheritance between meta-classes is to divide the matching process in two phases. First, we match the types which is the same as we did before, i.e., we are matching for equivalent types, and in a second step, for the remaining types, we look for subtype or supertype matches. However, for distinguishing between equivalent type matches and sub-/supertype matches, we weight the latter only as the half of equivalent type matches, i.e., their match weight is 0.5 instead of 1.

Let's consider the Family2Person example, where we have for Ri/Rj \manuel{add the exact example} the sets $C_{i}$ = \{ Family, Member, Person \} and $R_{j}$ = \{ Family, Female\} then the matching function has a full match for Family/Family, which weight is 1, and a supertype match for Person/Female with a weight of 0.5.

\subsubsection{Revised Matching Function}

To consider inheritance in the alignment process, we have to extend the matching function.
The new metrics are:

\begin{align}
CC_{i,j} = \dfrac{| C_{i} \cap_{inh} R_{j} |}{| C_{i} |}
\label{for:firstmetric}
\end{align}

\begin{align}
RC_{i,j} = \dfrac{| C_{i} \cap_{inh} R_{j} |}{| R_{j} |}
\label{for:secondmetric}
\end{align}

\begin{align}
RCR_{i,j} = \dfrac{| C_{i} \cap_{inh} R_{j} |}{| C_{i} \cup R_{j} |}
\label{for:thirdmetric}
\end{align}

where $ C \cap_{inh} R $ operation is given in the following. As can be seen in the listing, the first phase is used for checking the full matches and in the second phase the sub-/supertype matches are computed.

\begin{lstlisting}
Input: C, R
Output: vAbs
vAbs = 0;
for (Type c : C) do
    if (R.contains(c))
        vAbs++
        R.remove(c)
    endif
endfor
for (Type c : C) do
    subSuperType = R.getOneSubOrSuperType(subAndSuperTypes(c))
    if (subSuperType != null)
        vAbs=vAbs+0.5
        R.remove(subSuperType);
    endif
endfor
return vAbs
\end{lstlisting}

\begin{algorithm}[h]

% Data
%%%%%%%%%%%%%%%%%%%%%%%%%%
%%% Versions
\SetKwData{C}{C}
\SetKwData{cc}{c}
\SetKwData{R}{R}
\SetKwData{vAbs}{vAbs}
\SetKwData{subSuperType}{subSuperType}

% Keywords
%%%%%%%%%%%%%%%%%%%%%%%%%%%
\SetKw{Null}{null}
\SetKw{Return}{return}

% Functions
%%%%%%%%%%%%%%%%%%%%%%%%%%
\SetKwFunction{Remove}{remove}
\SetKwFunction{Contains}{contains}
\SetKwFunction{ContainsAny}{containsAny}


\KwIn{\C, \R}
\KwOut{\vAbs}

\BlankLine
\vAbs = 0
\tcp{Find full matches}
\For{\cc $\in$ \C}{
    \If{\R.\Contains{\cc}}{
    \vAbs = \vAbs + 1\\
    \R.\Remove{\cc}
    }
}
\tcp{Find sub-/supertype matches}
\For{\cc $\in$ \C}{
    \subSuperType $=$ \R.\ContainsAny{\subSuperType{\cc}}\\
    \If{\subSuperType $<>$ \Null}{
    \vAbs = \vAbs + 0.5\\
    \R.\Remove{\subSuperType}
    }
}
\Return \vAbs
\BlankLine
\caption{Matching Function extended with Inheritance}
\label{alg:inheritance}
\end{algorithm}


\subsubsection{Results}

Table \ref{tab:TableUML2ERInh} shows the results for the UML2ER example when the inheritance-aware metrics are computed.

\begin{table}[!h]
\centering
\caption{Matching table using the inheritance.}
\begin{tabular}{c|c|c|c|c|c|c|c|c||c|c|c|c|c|c|c|c||c|c|c|c|c|c|c|c|} \cline{2-25}
& \multicolumn{8}{|c||}{CC} & \multicolumn{8}{|c||}{RC} & \multicolumn{8}{|c|}{RCR}\\ \cline{2-25}
&R1&R2&R3&R4&R5&R6&R7&R8&R1&R2&R3&R4&R5&R6&R7&R8&R1&R2&R3&R4&R5&R6&R7&R8\\ \hline \hline
\multicolumn{1}{|c|}{C1}&0.5&1.0&0.0&0.0&0.0&0.0&0.0&0.0&0.5&1.0&0.0&0.0&0.0&0.0&0.0&0.0&0.5&1.0&0.0&0.0&0.0&0.0&0.0&0.0\\ \hline
\multicolumn{1}{|c|}{C2}&0.25&0.5&0.5&0.0&0.0&0.0&0.0&0.0&0.5&1.0&1.0&0.0&0.0&0.0&0.0&0.0&0.5&1.0&1.0&0.0&0.0&0.0&0.0&0.0\\ \hline
\multicolumn{1}{|c|}{C3}&0.16&0.33&0.33&0.33&0.25&0.25&0.25&0.25&0.5&1.0&1.0&1.0&0.75&0.75&0.75&0.75&0.5&1.0&1.0&1.0&0.75&0.75&0.75&0.75\\ \hline
\multicolumn{1}{|c|}{C4}&1.0&0.5&0.5&0.5&0.5&0.5&0.5&0.5&1.0&0.5&0.5&0.5&0.5&0.5&0.5&0.5&1.0&0.5&0.5&0.5&0.5&0.5&0.5&0.5\\ \hline
\multicolumn{1}{|c|}{C5}&0.5&1.0&0.0&0.0&0.0&0.0&0.0&0.0&0.5&1.0&0.0&0.0&0.0&0.0&0.0&0.0&0.5&1.0&0.0&0.0&0.0&0.0&0.0&0.0\\ \hline
\multicolumn{1}{|c|}{C6}&0.5&0.0&1.0&0.0&0.0&0.0&0.0&0.0&0.5&0.0&1.0&0.0&0.0&0.0&0.0&0.0&0.5&0.0&1.0&0.0&0.0&0.0&0.0&0.0\\ \hline
\multicolumn{1}{|c|}{C7}&0.5&0.0&0.0&1.0&0.75&0.75&0.75&0.75&0.5&0.0&0.0&1.0&0.75&0.75&0.75&0.75&0.5&0.0&0.0&1.0&0.75&0.75&0.75&0.75\\ \hline
\multicolumn{1}{|c|}{C8}&0.14&0.28&0.28&0.28&0.28&0.21&0.21&0.21&0.5&1.0&1.0&1.0&1.0&0.75&0.75&0.75&0.5&1.0&1.0&1.0&1.0&0.75&0.75&0.75\\ \hline
\multicolumn{1}{|c|}{C9}&0.12&0.25&0.25&0.25&0.18&0.25&0.25&0.18&0.5&1.0&1.0&1.0&0.75&1.0&1.0&0.75&0.5&1.0&1.0&1.0&0.75&1.0&1.0&0.75\\ \hline
\multicolumn{1}{|c|}{C10}&0.12&0.25&0.25&0.25&0.18&0.25&0.18&0.25&0.5&1.0&1.0&1.0&0.75&1.0&0.75&1.0&0.5&1.0&1.0&1.0&0.75&1.0&0.75&1.0\\ \hline
\end{tabular}
\label{tab:TableUML2ERInh}
\end{table}

As we can see in the table, ... positive effects/negative effects...

\subsection{Considering Features}

In this subsection we deal additional information extractable from tracts and ATL rules. Until now, we only extracted types from the constraints and the ATL rules. Now we also take care of the features used in tracts and ATL rules. For example, in the UML2ER transformation, rule 1 counts on two types (NamedElement and Element) and two features (NamedElement.name and Element.name). Now, we find a match between the constraint and rule sets when there are either two equal types in both or two equal features (notice that we are considering separately types and features and they do not match each other). This allows to explore the differences when using only features, only types, and their combination.

\subsubsection{Revised Matching Function}

In case features and types are considered separately or also combined, the matching function and the metrics are the same as we presented in Subsection~\ref{subsec:MatchingTables} with the only difference that in the input sets only features, only types, or both are considered. Anyway, we could consider the types and the features as two subsets and this the distinction is reflected in the pseudocode for the $ C \cap_{tf} R $ as follows:

\manuel{the two loops are more or less the same. can we generalize to element? maybe it makes also sense to present the Ecore metamodel in the tech report?}

\begin{lstlisting}
Input: C, R
vAbs = 0;
for (Type c : C) do
    if (R.contains(c))
        vAbs++
    endif
endfor
for (Feature f : C) do
    if (R.contains(f))
        vAbs++
    endif
endfor
return vAbs
\end{lstlisting}

\subsubsection{Results}

Table \ref{tab:TableUML2ERFeatures} shows the results and there we can see one expected property which is that the values will be always equal or lower than before, but all of them keep more or less the same relative distance among them. This means that we can use features or not, even we can think about use only use features and ignore the types, but the interpretation of the tables will be the same, so in certain way, we can choose what we want.

\manuel{this means, the table is presenting the metrics using types and features, right? Maybe it is good to show before the table only using features? Can we somehow compute the deviation between the results using types and the results using features? would be a nice approach to strengthen the discussion of the results. Maybe we can also conclude that features are not having a strong influence on the results, and reasoning on types may be sufficient...}

\begin{table}[!h]
\centering
\caption{Matching table considering features.}
\begin{tabular}{c|c|c|c|c|c|c|c|c||c|c|c|c|c|c|c|c||c|c|c|c|c|c|c|c|} \cline{2-25}
& \multicolumn{8}{|c||}{CC} & \multicolumn{8}{|c||}{RC} & \multicolumn{8}{|c|}{RCR}\\ \cline{2-25}
&R1&R2&R3&R4&R5&R6&R7&R8&R1&R2&R3&R4&R5&R6&R7&R8&R1&R2&R3&R4&R5&R6&R7&R8\\ \hline \hline
\multicolumn{1}{|c|}{C1}&0.0&0.5&0.0&0.0&0.0&0.0&0.0&0.0&0.0&0.5&0.0&0.0&0.0&0.0&0.0&0.0&0.0&0.33&0.0&0.0&0.0&0.0&0.0&0.0\\ \hline
\multicolumn{1}{|c|}{C2}&0.0&0.3&0.2&0.0&0.0&0.0&0.0&0.0&0.0&0.75&0.5&0.0&0.0&0.0&0.0&0.0&0.0&0.27&0.16&0.0&0.0&0.0&0.0&0.0\\ \hline
\multicolumn{1}{|c|}{C3}&0.0&0.13&0.2&0.13&0.06&0.06&0.06&0.06&0.0&0.5&0.75&1.0&0.25&0.25&0.33&0.33&0.0&0.11&0.18&0.13&0.05&0.05&0.05&0.05\\ \hline
\multicolumn{1}{|c|}{C4}&1.0&0.0&0.0&0.0&0.0&0.0&0.0&0.0&0.5&0.0&0.0&0.0&0.0&0.0&0.0&0.0&0.5&0.0&0.0&0.0&0.0&0.0&0.0&0.0\\ \hline
\multicolumn{1}{|c|}{C5}&0.0&1.0&0.0&0.0&0.0&0.0&0.0&0.0&0.0&0.5&0.0&0.0&0.0&0.0&0.0&0.0&0.0&0.5&0.0&0.0&0.0&0.0&0.0&0.0\\ \hline
\multicolumn{1}{|c|}{C6}&0.0&0.0&1.0&0.0&0.0&0.0&0.0&0.0&0.0&0.0&0.5&0.0&0.0&0.0&0.0&0.0&0.0&0.0&0.5&0.0&0.0&0.0&0.0&0.0\\ \hline
\multicolumn{1}{|c|}{C7}&0.0&0.0&0.0&1.0&0.5&0.5&0.5&0.5&0.0&0.0&0.0&1.0&0.25&0.25&0.33&0.33&0.0&0.0&0.0&1.0&0.2&0.2&0.25&0.25\\ \hline
\multicolumn{1}{|c|}{C8}&0.0&0.15&0.15&0.10&0.15&0.05&0.05&0.05&0.0&0.75&0.75&1.0&0.75&0.25&0.33&0.33&0.0&0.15&0.15&0.10&0.15&0.04&0.04&0.04\\ \hline
\multicolumn{1}{|c|}{C9}&0.0&0.14&0.14&0.09&0.04&0.14&0.14&0.09&0.0&0.75&0.75&1.0&0.25&0.75&1.0&0.66&0.0&0.13&0.13&0.09&0.04&0.13&0.14&0.09\\ \hline
\multicolumn{1}{|c|}{C10}&0.0&0.14&0.14&0.09&0.04&0.14&0.09&0.14&0.0&0.75&0.75&1.0&0.25&0.75&0.66&1.0&0.0&0.13&0.13&0.09&0.04&0.13&0.09&0.14\\ \hline
\end{tabular}
\label{tab:TableUML2ERFeatures}
\end{table}

\subsection{Considering Self-Information}

Until now we have considered that each type or feature match weight 1 (except for sub-/supertype matches when we consider inheritance between metaclasses where we use 0.5). However, when examining some constraint, we can find types or features which appear although they are not the key information. For instance, let us have a look to the constraint C10 in \emph{UML2ER} example. Some types (such as Property, ERModel, Class, EntityType, etc.) appear because of the nesting but the highlighted types should be Reference and StrongReference because they are exactly the core of the constraint. C10 does not aim to put restrictions for Property, but this type is necessary to define the constraint correctly. Trying to deal with this, we have analyzed the types used in contracts and modified the weights for the matching function taking into account information theory \cite{Shannon2001}: the amount of self-information contained in a probabilistic event depends only on the probability of that event: the smaller its probability, the larger the self-information associated with receiving the information that the event indeed occurred.

When adapting the notion of information theory to our problem domain, the end up with the following definition: the smaller the probability that a type occurs in a tract, the larger the self-information associated with receiving the information that the type indeed occurs in the tract. Thus, we employ the standard formula for computing self-information \emph{I} for a given model element:

\begin{align}
I(m) = - log2[p(m)]
\end{align}

where m is a model element and  p(m) is the probability of the model element m occurring in a tract.

To make things clear, we enumerate in the following table the occurrences for each type in the tracts, the probability, and the self-information:

\begin{table}[t]
\centering
\caption{Self-information of types for UML2ER example.}
\begin{tabular}{|c|l|l|l|} \hline
Type&Occurrences&Probability&Self-information\\ \hline
Package & 	7 &	0.7	&0.51 \\ \hline
ERModel & 	7 &	0.7	&0.51 \\ \hline
Class & 	6 &	0.6	&0.74 \\ \hline
EntityType & 	6	& 0.6	& 0.74 \\ \hline
Property &	5	& 0.5	& 1 \\ \hline
Feature &	5	&0.5	& 1 \\ \hline
Attribute &	1	&0.1	& 3.32 \\ \hline
Reference & 	2	&0.2	& 2.32 \\ \hline
WeakReference & 	1	&0.1	& 3.32 \\ \hline
StrongReference & 	1	&0.1	& 3.32\\ \hline
\end{tabular}
\label{tab:SelfInformation}
\end{table}

As we can see in the table, some types occur quite frequently (around 0.7) whereas other are rather limited occurring (around 0.10). This gives us quite different numbers for the calculated self-information. If we now want to consider the self-information for computing the alignments, we have to consider it in the matching function.

\subsubsection{Revised Matching Function}

In general, we may distinguish between basic tracts consisting of only a few set of types and tracts containing several types needed for formulating the constraint. To give you an idea how the distribution of the type size is for the UML2ER example, the following table contains the numbers of types and features for each of the ten tracts.

\begin{table}[t]
\centering
\caption{Self-information of types for UML2ER example.}
\begin{tabular}{|c|l|l|} \hline
Tract&Used Types&Used Features\\ \hline
C1	& 2	& 2\\ \hline
C2	& 4	& 6\\ \hline
C3	& 6	& 10\\ \hline
C4	& 2	& 0\\ \hline
C5	& 2	& 0\\ \hline
C6	& 2	& 0\\ \hline
C7	& 2	& 0\\ \hline
C8	& 7	& 12\\ \hline
C9	& 8	& 13\\ \hline
C10	& 8	& 13\\ \hline\hline
Avg	& 4.3 & 5.6\\\hline
\end{tabular}
\label{tab:Family2PersonExample}
\end{table}


\begin{lstlisting}
Input: C, R, avgTypeSize
Output: vAbs
vAbs = 0;
for (Type c : C) do
    if (R.contains(c))
        if (C.size() > avgTypeSize)						
	       if(c.selfInformation() <= 0,25)
                continue					
	       else if(c.selfInformation < 1)
                vAbs = vAbs + c.selfinformation - 0,25					
	       else
                vAbs = vAbs + 1	
        endif		
        R.remove(c)
    endif
endfor
return vAbs
\end{lstlisting}

\subsubsection{Results}

The results considering inheritance for types and weighting them with the previous procedure are shown in table \ref{tab:TableUML2ERSelfInfo}.

\begin{table}[!h]
\centering
\caption{Matching table considering inheritance and self-information.}
\begin{tabular}{c|c|c|c|c|c|c|c|c||c|c|c|c|c|c|c|c||c|c|c|c|c|c|c|c|} \cline{2-25}
& \multicolumn{8}{|c||}{CC} & \multicolumn{8}{|c||}{RC} & \multicolumn{8}{|c|}{RCR}\\ \cline{2-25}
&R1&R2&R3&R4&R5&R6&R7&R8&R1&R2&R3&R4&R5&R6&R7&R8&R1&R2&R3&R4&R5&R6&R7&R8\\ \hline \hline
\multicolumn{1}{|c|}{C1}&0.5&1.0&0.0&0.0&0.0&0.0&0.0&0.0&0.5&1.0&0.0&0.0&0.0&0.0&0.0&0.0&0.5&1.0&0.0&0.0&0.0&0.0&0.0&0.0\\ \hline
\multicolumn{1}{|c|}{C2}&0.25&0.5&0.5&0.0&0.0&0.0&0.0&0.0&0.5&1.0&1.0&0.0&0.0&0.0&0.0&0.0& 0.5&1.0&1.0&0.0&0.0&0.0&0.0&0.0\\ \hline
\multicolumn{1}{|c|}{C3}&0.04&0.09&0.14&0.20&0.15&0.15&0.15&0.15&0.14&0.28&0.44&0.62&0.46&0.46&0.46&0.46&0.14&0.28&0.44&0.62&0.46&0.46&0.46&0.46\\ \hline
\multicolumn{1}{|c|}{C4}&1.0&0.5&0.5&0.5&0.5&0.5&0.5&0.5&1.0&0.5&0.5&0.5&0.5&0.5&0.5&0.5&1.0&0.5&0.5&0.5&0.5&0.5&0.5&0.5\\ \hline
\multicolumn{1}{|c|}{C5}&0.5&1.0&0.0&0.0&0.0&0.0&0.0&0.0&0.5&1.0&0.0&0.0&0.0&0.0&0.0&0.0&0.5&1.0&0.0&0.0&0.0&0.0&0.0&0.0\\ \hline
\multicolumn{1}{|c|}{C6}&0.5&0.0&1.0&0.0&0.0&0.0&0.0&0.0&0.5&0.0&1.0&0.0&0.0&0.0&0.0&0.0&0.5&0.0&1.0&0.0&0.0&0.0&0.0&0.0\\ \hline
\multicolumn{1}{|c|}{C7}&0.5&0.0&0.0&1.0&0.75&0.75&0.75&0.75&0.5&0.0&0.0&1.0&0.75&0.75&0.75&0.75&0.5&0.0&0.0&1.0&0.75&0.75&0.75&0.75\\ \hline
\multicolumn{1}{|c|}{C8}&0.04&0.08&0.12&0.17&0.23&0.13&0.13&0.13&0.14&0.28&0.44&0.62&0.81&0.46&0.46&0.46&0.14&0.28&0.44&0.62&0.81&0.46&0.46&0.46\\ \hline
\multicolumn{1}{|c|}{C9}&0.03&0.07&0.11&0.15&0.11&0.20&0.20&0.11&0.14&0.28&0.44&0.62&0.46&0.81&0.81&0.46&0.14&0.28&0.44&0.62&0.46&0.81&0.81&0.46\\ \hline
\multicolumn{1}{|c|}{C10}&0.03&0.07&0.11&0.15&0.11&0.20&0.11&0.20&0.14&0.28&0.44&0.62&0.46&0.81&0.46&0.81&0.14&0.28&0.44&0.62&0.46&0.81&0.46&0.81\\ \hline
\end{tabular}
\label{tab:TableUML2ERSelfInfo}
\end{table}

\subsection{Combined Configurable Matching Function}

\subsubsection{Revised Matching Function}

\subsubsection{Results}



%%%%%%%%%%%%%%%%%%%%%%%%%%%%%%%%%%%%%%%%%%%%%%%%%%%%%%%%%%%%%%%%%%%%%%%%%%%%%%%%%%%%%%%%%%%%%%%%%%%%%%%%%%%%%%%%%%%%%%%%%%%%%%%%%%%%%%%%%




\section{Implementation}
\label{sec:Implementation}

\begin{figure}[h!]
\centering
\includegraphics[width=200pt]{images/Process}
\caption{Matching process at a glance.}
\label{fig:Process}
\end{figure}

In order to obtain the result shown in the previous subsection, it is beneficial to have automation support for the matching process. Figure~\ref{fig:Process} depicts each step of the matching process. The initial input for this processes are the constraints and the transformation rules. The output are the matching tables as explained before.

Starting with the constraint branch, the first step is to extract the types for each constraint. This is achieved by employing the API of the USE (UML based Specification Environment) tool \cite{USE2}. This API allows to parse an OCL expression and provides the parsing results in a model-based representation. Using this representation, we are able to extract all the types used within an OCL expression. This is actually provided by having the parse tree representing each subexpression by an explicit node which also provides the return type for each subexpression.

The types extraction for ATL transformations is more challenging compared to the OCL part, because currently there is no support offered by the ATL implementation. However, the textual ATL transformations can be automatically injected to model-based representations. This model-based representation allows to extract the needed information from an ATL transformation by applying another ATL transformation (a so-called higher-order transformation) which generates a model stating for each ATL rule all used types of the input and output pattern elements. Currently, we only support to extract the explicitly given types. The extraction of implicit types used in filter and binding expressions is subject to future work. %\manuel{can we use USE to give us the types for OCL expressions used in ATL code?}

Having the used types of all constraints and rules, we may apply the matching functions---which coincide with the metrics described in the previous subsection. The matching functions are implemented in Java and the output of the computation is either represented as an Excel file as well as can be exported as an EMF-based model for further computations, e.g., by applying further model transformations for analysing the matching tables. The \emph{Tracts2ATL} Matcher prototype can be downloaded from our project website\footnote{\url{http://atenea.lcc.uma.es/Descargas/Tract2ATLMatcher.zip}}. 


\section{Related Work}
\label{sec:RelatedWork}
With respect to the contribution of this paper, two threads of related work are discussed.
First, there are general traceability approaches in software engineering, and second, there are specific approaches for tracking ``guilty'' transformation rules, i.e., those whose behaviour violate the transformation specifications.

IEEE \cite{IEEE90} defines traceability as the degree to which a relationship between two or more artifacts can be established. Most tracing approaches are dedicated to establish traceability links between artifacts that are in a predecessor/successor relationship with respect to their creation time, e.g., between requirements, features, design, architecture, and code. Our approach for automatically finding the alignments between constraints and transformation rules are in the spirit of traceability rules as presented in \cite{RameshD92,PinheiroG96}. A survey dedicated to traceability in the field of MDE is presented in \cite{GalvaoG07}, where the possibilities of using trace links established by model transformations are discussed. However, this survey does not report on tracing approaches between transformation specifications and implementations.

Tracking guilty transformation rules using a dynamic approach, i.e., by executing the model transformation under test, has been subject to investigations. In \cite{Wimmer09}, we used OCL-based queries to backwards debugging of model transformations using an explicit runtime model based on the trace model between the source and target models. Aranega et al. \cite{AranegaMED09} present an approach for situating transformations errors by exploiting also the traces between the source and target models. The dynamic approach is also used by \cite{UjhelyiHV12} to build slices of model transformations. While these approaches are all tracking transformation rules using specific test input models, our aim is to statically build more general traceability models between transformations' specifications and their implementations. In \cite{Jeanneret11}, model footprints of operations are statically computed by the use of metamodel footprints. We pursue the idea of computing metamodel footprints from transformation specifications and implementations for establishing traceability links instead of reasoning on model footprints.



\section{Next Steps}
\label{sec:NextSteps}
The main motivation for this work was the need to track transformation rules that can be considered ``guilty'' for violating parts of the transformation specifications. Due to the generic nature of the matching tables, a multitude of further use cases emerge.

\textbf{Properties of Alignments.} Based on the matching tables, we are able to reason on the degree of tangling and scattering between constraints and rules. Scattering occurs when a single constraint is scattered across multiple rules, while tangling occurs when a single transformation rule is implementing multiple constraints at once. The work of Berg et al.~\cite{BergCH07} may be valid input to reason about design guidelines of transformation specifications and implementations based on matching tables.

\textbf{Refinement of Alignments.} More information may be extracted from constraints and transformation rules. For example, from the ATL transformations, inheritance between transformation rules, lazy rule calls, and types used in filters and bindings may be extracted. From the constraints, the accessed metamodel features may be extracted, too. Based on this additional information, more refined alignments may be explored. Furthermore, as we have mentioned in the evaluation, some constraints are using a multitude of types. To distinguish between types, e.g., types only required to navigate to the most relevant information in a model, types occurring more often in constraints may have less impact on the alignments as types that do not as frequently occur.

\textbf{Alignment-based Slicing.} Another direction for future work is to slice model transformations, metamodels, and models based on constraints. This is of course useful for debugging model transformations, however, using slicing techniques may be also beneficial for maintenance tasks. Imagine the requirements are changed by modifying a specific constraint. Adapting the transformation implementation to this change may be easier by reasoning only on a particular slice of the transformation problem referring to a subset of the transformation, metamodel, and models.


\bibliographystyle{abbrv}
\bibliography{TestingMT}

\end{document}

%\section{The {\secit Body} of The Paper}
%Typically, the body of a paper is organized
%into a hierarchical structure, with numbered or unnumbered
%headings for sections, subsections, sub-subsections, and even
%smaller sections.  The command \texttt{{\char'134}section} that
%precedes this paragraph is part of such a
%hierarchy.\footnote{This is the second footnote.  It
%starts a series of three footnotes that add nothing
%informational, but just give an idea of how footnotes work
%and look. It is a wordy one, just so you see
%how a longish one plays out.} \LaTeX\ handles the numbering
%and placement of these headings for you, when you use
%the appropriate heading commands around the titles
%of the headings.  If you want a sub-subsection or
%smaller part to be unnumbered in your output, simply append an
%asterisk to the command name.  Examples of both
%numbered and unnumbered headings will appear throughout the
%balance of this sample document.
%
%Because the entire article is contained in
%the \textbf{document} environment, you can indicate the
%start of a new paragraph with a blank line in your
%input file; that is why this sentence forms a separate paragraph.
%
%\subsection{Type Changes and {\subsecit Special} Characters}
%We have already seen several typeface changes in this sample.  You
%can indicate italicized words or phrases in your text with
%the command \texttt{{\char'134}textit}; emboldening with the
%command \texttt{{\char'134}textbf}
%and typewriter-style (for instance, for computer code) with
%\texttt{{\char'134}texttt}.  But remember, you do not
%have to indicate typestyle changes when such changes are
%part of the \textit{structural} elements of your
%article; for instance, the heading of this subsection will
%be in a sans serif\footnote{A third footnote, here.
%Let's make this a rather short one to
%see how it looks.} typeface, but that is handled by the
%document class file. Take care with the use
%of\footnote{A fourth, and last, footnote.}
%the curly braces in typeface changes; they mark
%the beginning and end of
%the text that is to be in the different typeface.
%
%You can use whatever symbols, accented characters, or
%non-English characters you need anywhere in your document;
%you can find a complete list of what is
%available in the \textit{\LaTeX\
%User's Guide}\cite{Lamport:LaTeX}.
%
%\subsection{Math Equations}
%You may want to display math equations in three distinct styles:
%inline, numbered or non-numbered display.  Each of
%the three are discussed in the next sections.
%
%\subsubsection{Inline (In-text) Equations}
%A formula that appears in the running text is called an
%inline or in-text formula.  It is produced by the
%\textbf{math} environment, which can be
%invoked with the usual \texttt{{\char'134}begin. . .{\char'134}end}
%construction or with the short form \texttt{\$. . .\$}. You
%can use any of the symbols and structures,
%from $\alpha$ to $\omega$, available in
%\LaTeX\cite{Lamport:LaTeX}; this section will simply show a
%few examples of in-text equations in context. Notice how
%this equation: \begin{math}\lim_{n\rightarrow \infty}x=0\end{math},
%set here in in-line math style, looks slightly different when
%set in display style.  (See next section).
%
%\subsubsection{Display Equations}
%A numbered display equation -- one set off by vertical space
%from the text and centered horizontally -- is produced
%by the \textbf{equation} environment. An unnumbered display
%equation is produced by the \textbf{displaymath} environment.
%
%Again, in either environment, you can use any of the symbols
%and structures available in \LaTeX; this section will just
%give a couple of examples of display equations in context.
%First, consider the equation, shown as an inline equation above:
%\begin{equation}\lim_{n\rightarrow \infty}x=0\end{equation}
%Notice how it is formatted somewhat differently in
%the \textbf{displaymath}
%environment.  Now, we'll enter an unnumbered equation:
%\begin{displaymath}\sum_{i=0}^{\infty} x + 1\end{displaymath}
%and follow it with another numbered equation:
%\begin{equation}\sum_{i=0}^{\infty}x_i=\int_{0}^{\pi+2} f\end{equation}
%just to demonstrate \LaTeX's able handling of numbering.
%
%\subsection{Citations}
%Citations to articles \cite{bowman:reasoning, clark:pct, braams:babel, herlihy:methodology},
%conference
%proceedings \cite{clark:pct} or books \cite{salas:calculus, Lamport:LaTeX} listed
%in the Bibliography section of your
%article will occur throughout the text of your article.
%You should use BibTeX to automatically produce this bibliography;
%you simply need to insert one of several citation commands with
%a key of the item cited in the proper location in
%the \texttt{.tex} file \cite{Lamport:LaTeX}.
%The key is a short reference you invent to uniquely
%identify each work; in this sample document, the key is
%the first author's surname and a
%word from the title.  This identifying key is included
%with each item in the \texttt{.bib} file for your article.
%
%The details of the construction of the \texttt{.bib} file
%are beyond the scope of this sample document, but more
%information can be found in the \textit{Author's Guide},
%and exhaustive details in the \textit{\LaTeX\ User's
%Guide}\cite{Lamport:LaTeX}.
%
%This article shows only the plainest form
%of the citation command, using \texttt{{\char'134}cite}.
%This is what is stipulated in the SIGS style specifications.
%No other citation format is endorsed.
%
%\subsection{Tables}
%Because tables cannot be split across pages, the best
%placement for them is typically the top of the page
%nearest their initial cite.  To
%ensure this proper ``floating'' placement of tables, use the
%environment \textbf{table} to enclose the table's contents and
%the table caption.  The contents of the table itself must go
%in the \textbf{tabular} environment, to
%be aligned properly in rows and columns, with the desired
%horizontal and vertical rules.  Again, detailed instructions
%on \textbf{tabular} material
%is found in the \textit{\LaTeX\ User's Guide}.
%
%Immediately following this sentence is the point at which
%Table 1 is included in the input file; compare the
%placement of the table here with the table in the printed
%dvi output of this document.
%
%\begin{table}
%\centering
%\caption{Frequency of Special Characters}
%\begin{tabular}{|c|c|l|} \hline
%Non-English or Math&Frequency&Comments\\ \hline
%\O & 1 in 1,000& For Swedish names\\ \hline
%$\pi$ & 1 in 5& Common in math\\ \hline
%\$ & 4 in 5 & Used in business\\ \hline
%$\Psi^2_1$ & 1 in 40,000& Unexplained usage\\
%\hline\end{tabular}
%\end{table}
%
%To set a wider table, which takes up the whole width of
%the page's live area, use the environment
%\textbf{table*} to enclose the table's contents and
%the table caption.  As with a single-column table, this wide
%table will ``float" to a location deemed more desirable.
%Immediately following this sentence is the point at which
%Table 2 is included in the input file; again, it is
%instructive to compare the placement of the
%table here with the table in the printed dvi
%output of this document.
%
%
%\begin{table*}
%\centering
%\caption{Some Typical Commands}
%\begin{tabular}{|c|c|l|} \hline
%Command&A Number&Comments\\ \hline
%\texttt{{\char'134}alignauthor} & 100& Author alignment\\ \hline
%\texttt{{\char'134}numberofauthors}& 200& Author enumeration\\ \hline
%\texttt{{\char'134}table}& 300 & For tables\\ \hline
%\texttt{{\char'134}table*}& 400& For wider tables\\ \hline\end{tabular}
%\end{table*}
%% end the environment with {table*}, NOTE not {table}!
%
%\subsection{Figures}
%Like tables, figures cannot be split across pages; the
%best placement for them
%is typically the top or the bottom of the page nearest
%their initial cite.  To ensure this proper ``floating'' placement
%of figures, use the environment
%\textbf{figure} to enclose the figure and its caption.
%
%This sample document contains examples of \textbf{.eps}
%and \textbf{.ps} files to be displayable with \LaTeX.  More
%details on each of these is found in the \textit{Author's Guide}.
%
%%\begin{figure}
%%\centering
%%\epsfig{file=fly.eps}
%%\caption{A sample black and white graphic (.eps format).}
%%\end{figure}
%%
%%\begin{figure}
%%\centering
%%\epsfig{file=fly.eps, height=1in, width=1in}
%%\caption{A sample black and white graphic (.eps format)
%%that has been resized with the \texttt{epsfig} command.}
%%\end{figure}
%
%
%As was the case with tables, you may want a figure
%that spans two columns.  To do this, and still to
%ensure proper ``floating'' placement of tables, use the environment
%\textbf{figure*} to enclose the figure and its caption.
%
%Note that either {\textbf{.ps}} or {\textbf{.eps}} formats are
%used; use
%the \texttt{{\char'134}epsfig} or \texttt{{\char'134}psfig}
%commands as appropriate for the different file types.
%
%\subsection{Theorem-like Constructs}
%Other common constructs that may occur in your article are
%the forms for logical constructs like theorems, axioms,
%corollaries and proofs.  There are
%two forms, one produced by the
%command \texttt{{\char'134}newtheorem} and the
%other by the command \texttt{{\char'134}newdef}; perhaps
%the clearest and easiest way to distinguish them is
%to compare the two in the output of this sample document:
%
%This uses the \textbf{theorem} environment, created by
%the\linebreak\texttt{{\char'134}newtheorem} command:
%\newtheorem{theorem}{Theorem}
%\begin{theorem}
%Let $f$ be continuous on $[a,b]$.  If $G$ is
%an antiderivative for $f$ on $[a,b]$, then
%\begin{displaymath}\int^b_af(t)dt = G(b) - G(a).\end{displaymath}
%\end{theorem}
%
%The other uses the \textbf{definition} environment, created
%by the \texttt{{\char'134}newdef} command:
%\newdef{definition}{Definition}
%\begin{definition}
%If $z$ is irrational, then by $e^z$ we mean the
%unique number which has
%logarithm $z$: \begin{displaymath}{\log e^z = z}\end{displaymath}
%\end{definition}
%
%%\begin{figure}
%%\centering
%%\psfig{file=rosette.ps, height=1in, width=1in,}
%%\caption{A sample black and white graphic (.ps format) that has
%%been resized with the \texttt{psfig} command.}
%%\end{figure}
%
%Two lists of constructs that use one of these
%forms is given in the
%\textit{Author's  Guidelines}.
%
%%\begin{figure*}
%%\centering
%%\epsfig{file=flies.eps}
%%\caption{A sample black and white graphic (.eps format)
%%that needs to span two columns of text.}
%%\end{figure*}
%and don't forget to end the environment with
%{figure*}, not {figure}!
%
%There is one other similar construct environment, which is
%already set up
%for you; i.e. you must \textit{not} use
%a \texttt{{\char'134}newdef} command to
%create it: the \textbf{proof} environment.  Here
%is a example of its use:
%\begin{proof}
%Suppose on the contrary there exists a real number $L$ such that
%\begin{displaymath}
%\lim_{x\rightarrow\infty} \frac{f(x)}{g(x)} = L.
%\end{displaymath}
%Then
%\begin{displaymath}
%l=\lim_{x\rightarrow c} f(x)
%= \lim_{x\rightarrow c}
%\left[ g{x} \cdot \frac{f(x)}{g(x)} \right ]
%= \lim_{x\rightarrow c} g(x) \cdot \lim_{x\rightarrow c}
%\frac{f(x)}{g(x)} = 0\cdot L = 0,
%\end{displaymath}
%which contradicts our assumption that $l\neq 0$.
%\end{proof}
%
%Complete rules about using these environments and using the
%two different creation commands are in the
%\textit{Author's Guide}; please consult it for more
%detailed instructions.  If you need to use another construct,
%not listed therein, which you want to have the same
%formatting as the Theorem
%or the Definition\cite{salas:calculus} shown above,
%use the \texttt{{\char'134}newtheorem} or the
%\texttt{{\char'134}newdef} command,
%respectively, to create it.
%
%\subsection*{A {\secit Caveat} for the \TeX\ Expert}
%Because you have just been given permission to
%use the \texttt{{\char'134}newdef} command to create a
%new form, you might think you can
%use \TeX's \texttt{{\char'134}def} to create a
%new command: \textit{Please refrain from doing this!}
%Remember that your \LaTeX\ source code is primarily intended
%to create camera-ready copy, but may be converted
%to other forms -- e.g. HTML. If you inadvertently omit
%some or all of the \texttt{{\char'134}def}s recompilation will
%be, to say the least, problematic.
%
%\section{Conclusions}
%This paragraph will end the body of this sample document.
%Remember that you might still have Acknowledgments or
%Appendices; brief samples of these
%follow.  There is still the Bibliography to deal with; and
%we will make a disclaimer about that here: with the exception
%of the reference to the \LaTeX\ book, the citations in
%this paper are to articles which have nothing to
%do with the present subject and are used as
%examples only.
%%\end{document}  % This is where a 'short' article might terminate
%
%%ACKNOWLEDGMENTS are optional
%\section{Acknowledgments}
%This section is optional; it is a location for you
%to acknowledge grants, funding, editing assistance and
%what have you.  In the present case, for example, the
%authors would like to thank Gerald Murray of ACM for
%his help in codifying this \textit{Author's Guide}
%and the \textbf{.cls} and \textbf{.tex} files that it describes.
%
%%
%% The following two commands are all you need in the
%% initial runs of your .tex file to
%% produce the bibliography for the citations in your paper.
%\bibliographystyle{abbrv}
%\bibliography{sigproc}  % sigproc.bib is the name of the Bibliography in this case
%% You must have a proper ".bib" file
%%  and remember to run:
%% latex bibtex latex latex
%% to resolve all references
%%
%% ACM needs 'a single self-contained file'!
%%
%%APPENDICES are optional
%%\balancecolumns
%\appendix
%%Appendix A
%\section{Headings in Appendices}
%The rules about hierarchical headings discussed above for
%the body of the article are different in the appendices.
%In the \textbf{appendix} environment, the command
%\textbf{section} is used to
%indicate the start of each Appendix, with alphabetic order
%designation (i.e. the first is A, the second B, etc.) and
%a title (if you include one).  So, if you need
%hierarchical structure
%\textit{within} an Appendix, start with \textbf{subsection} as the
%highest level. Here is an outline of the body of this
%document in Appendix-appropriate form:
%\subsection{Introduction}
%\subsection{The Body of the Paper}
%\subsubsection{Type Changes and  Special Characters}
%\subsubsection{Math Equations}
%\paragraph{Inline (In-text) Equations}
%\paragraph{Display Equations}
%\subsubsection{Citations}
%\subsubsection{Tables}
%\subsubsection{Figures}
%\subsubsection{Theorem-like Constructs}
%\subsubsection*{A Caveat for the \TeX\ Expert}
% That's all folks!
